% main.tex — THOR-FP ArXiv Submission (Bloc 30)
% Auteur : Alexandre ICHAÏ (www.thor.love)

\documentclass[11pt]{article}
\usepackage[utf8]{inputenc}
\usepackage{amsmath, amssymb}
\usepackage{graphicx}
\usepackage{hyperref}
\usepackage{geometry}
\geometry{margin=2.5cm}

\title{THOR-FP: Unified Scalar Topological Physics (\(\theta_0 = 42\pi\))}
\author{Alexandre ICHAÏ \\ \href{https://www.thor.love}{www.thor.love}}
\date{\today}

\begin{document}

\maketitle

\begin{abstract}
We introduce the THOR-FP model: a unified, falsifiable scalar topological law (\(\theta_0 = 42\pi\)) linking fractal harmonics, cosmology, quantum physics, and biology. This submission includes reproducible code, benchmark tables, and all figures as required by open science standards.
\end{abstract}

\section{Introduction}

% (À compléter avec le texte scientifique principal)
The THOR-FP (Topological Harmonic Oscillatory Resonance — Fractal Physics) framework proposes...

\section{Main Equation and Model}

The scalar invariant is defined as:
\[
\theta_0 = 42\pi \cdot \cos(\theta)
\]

% Ajoutez ici les équations détaillées, démonstrations, explications

\section{Methods}

% Présentez les méthodes, les codes, l'approche reproducible

\section{Results and Figures}

All key figures are available in the repository (/figures/).

\begin{figure}[ht]
    \centering
    \includegraphics[width=0.5\textwidth]{figures/fig1_dummy.png}
    \caption{Dummy Figure 1: Scalar Topological Invariant.}
\end{figure}

% Ajoutez d'autres figures ici

\section{Benchmarks}

% Intégrer ou référencer le tableau benchmark_table.csv

\section{Conclusion and Perspectives}

% Conclusion et perspectives

\section*{References}

\bibliographystyle{plain}
\bibliography{bib}

\end{document}
